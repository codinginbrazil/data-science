\subsection{Análise de grupos}

Grupos naturais descrito por Carmichael que grupos são aqueles que satisfazem duas condições particulares:
\begin{enumerate}
    \item Existência de regiões contínuas do espaço, relativamente densamente populadas por objetos;
    \item Tais regiões estão rodeadas por regiões relativamente vazias.
\end{enumerate}


\subsubsection{Medidas de similaridade}

Matriz de confusão (contingência)

\paragraph*{Dados binários}

Distância Hamming

\begin{table}
    \centering
    \caption{Medidas de dissimilaridade para variáveis contínuas}
    \begin{tabular}{l|l}
        \hline
        \multicolumn{1}{c|}{\textbf{Medida}} & \multicolumn{1}{c}{\textbf{Fórmula}}       \\ \hline
        S1: Coeficiente de Matching          & $ S_{ij} = \frac{a + d}{a + b + c + d}   $ \\ \hline
        S2: Coeficiente de Jaccard           & $ S_{ij} = \frac{a}{a+b+c}               $ \\ \hline
        S3: Rogers \& Tanimoto               & $ S_{ij} = \frac{a+d}{a+2(b+c)+d}        $ \\ \hline
        S4: Sokal \& Sneath                  & $ S_{ij} = \frac{a}{a+2(b+c)}            $ \\ \hline
        S5: Gower \& Legendre                & $ S_{ij} = \frac{a+d}{a+0.5(b+c)+d}      $ \\ \hline
        S6: Gower \& Legendre 2              & $ S_{ij} = \frac{a}{a+0.5(b+c)}          $ \\ \hline
    \end{tabular}
\end{table}


\subsubsection{Medidas de dissimilaridade para variáveis contínuas}

\paragraph*{Medidade distância}
\subparagraph*{Família de distância Minkowski}
\subparagraph*{Distância de Canberra}

\paragraph*{Medidas tipos de correlação}

\subparagraph*{Correlação de Pearson $ [-1,1] $}
\subparagraph*{Medidda do Cosseno $ [-1,1] $}


\subsubsection{Métodos de agrupamento}

\begin{itemize}
    \item Hierárquicos
    \item Particionais
\end{itemize}


\paragraph*{Avaliação}

\begin{itemize}
    \item Compactação
    \item Separação
\end{itemize}


Medidas internas, seguindo o índice de (p. 241):

\begin{itemize}
    \item Dunn \cite{dunn1973fuzzy}
    \item Davies-Bouldi [0; infinito] \cite{Davies_Bouldin}
    \item Bezdek-Pal \cite{bezdek_pal1998}
    \item Silhueta \cite{rousseeuw1987silhouettes}
\end{itemize}

Medidas externas

\begin{itemize}
    \item \underline{Entropia:} define homogeneidade dos grupos encontrados.
    Portanto, o valor de baixa entropia indica mais homogeneidade
    \item Pureza
    \item índice FBCubed \cite{amigo2009_FBCubed}
\end{itemize}