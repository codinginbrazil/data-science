\subsection{Pré-processamento de dados}

O processo de preparação da base de dados:

\begin{itemize}
	\item \underline{Limpeza de dados:}
		Imputação de valores ausentes, remoção de ruídos e correção de inconsistências;
	\item \underline{Integração dos dados:}
		Unir dados de múltiplas fontes em um único local, como um armazém de dados (data warehouse);
	\item \underline{Redução dos dados:}
		Reduzir a dimensão da base de dados, por exemplo, agrupando ou eliminando atributos redundantes, ou para reduzir a quantidade de objetos da base, sumarizando os dados;
	\item \underline{Transformação dos dados:}
		Padronizar e deixar os dados em um formato passível de aplicação das diferentes técnicas de mineração;
	\item \underline{ Discretização dos dados:}
		Permitir que métodos que trabalham apenas com atributos nominais possam ser empregados a um conjunto maior de problemas.
		Também faz com que a quantidade de valores para um dado atributo (contínuo) seja reduzida.
\end{itemize}


\textbf{Limpeza de dados:}
A baixa qualidade dos dados é um problema que afeta a maior parte das bases de dados reais.
Assim, as ferramentas para a limpeza de dados atuam no sentido de imputar valores ausentes, suavizar ruídos, identificar valores discrepantes (outliers) e corrigir inconsistências.


\textbf{Métodos tradicionais de imputação de valores ausentes:}
\begin{itemize}
	\item \underline{Avestruz:}
		descarta o objeto que possui atributo ausente.
	\item \underline{Manual:}
		escolher manual de forma empírica um valor a ser imputado para cada valor ausente.
	\item \underline{Constante:}
		substitui todo valor ausente por uma constante.
	\item \underline{Hot-deck:}
		substitui o valor ausente por um valor mais similar a ele.
	\item \underline{Last observation carried forward:}
		considera que a representação é uma medida contínua, para isto ordena todos os atributos, substituindo os valores ausentes por seus antecessores.
	\item \underline{Medidas centrais:}
		usar a média ou a moda para substituir valores ausentes.
	\item \underline{Medidas centrais para classe:}
		usar a média ou a moda da classe para substituir valores ausentes da mesma.
	\item \underline{Modelo preditivos:}
		utiliza modelo preditivos para imputar os valores ausentes.
		Nesse caso, o atributo com valores ausentes é utilizado como atributo dependente, ao passo que os outros atributos são usados como independentes para se criar o modelo preditivo.
		Portanto, o modelo preditivo é usado para estimar os valores ausentes.
\end{itemize}


\textbf{Métodos de Redução de dados}
\begin{itemize}
	\item \underline{Redução de dimensionalidade:}
		seleção de atributos
	\item \underline{Compressão de atributos:}
		também efetua uma redução da dimensionalidade, mas empregando algoritmos de codificação ou transformação de dados (atributos), em vez de seleção.
		Exemplo é a Análise de Componentes Principais~(Principal Component Analysis – PCA), que é um procedimento estatístico que converte um conjunto de objetos com atributos possivelmente correlacionados em um conjunto de objetos com atributos linearmente descorrelacionados, chamados de componentes principais.
		O número de componentes principais é menor ou igual ao número de atributos da base, e a transformação é definida de forma que o primeiro componente principal possua a maior variância (ou seja, represente a maior variabilidade dos dados), o segundo componente principal
	\item \underline{Redução de número de dados:}
		realiza um corte temporal das instância, podendo ser combinada com a redução de dimensionalidade.
	\item \underline{Discretização:}
		os valores de atributos são substituídos por intervalos ou níveis conceituais mais elevados, reduzindo a quantidade final de atributos.
\end{itemize}


\textbf{Transformação dos dados}

Padronização: escala e unidades em bases compatíveis.

Normalização

\begin{itemize}
	\item \underline{Máximo pelo minímo:}
		A normalização Max-Min realiza uma transformação linear nos dados originais. Assuma que $max_a$ e $min_a$ são,respectivamente, os valores máximo e mínimo de determinado atributo $a$.
		A normalização max-min mapeia um valor a em um valor $a'$ no domínio $[novo_min_a', novo_max_a']$, de acordo com a Equação abaixo.
		A aplicação mais frequente dessa normalização é colocar todos os atributos de uma base de dados sob um mesmo intervalo de valores, por exemplo no intervalo $[0, 1]$.
		\begin{equation} \label{eq:max_min}
			a' = \frac{a - min_a}{max_a - min_a}
		\end{equation}
	\item \underline{Escore-Z (Escore Padronizado):}
		Útil quando se desconhece a amplitude dos dados ou há outliers, faz parte das medidas de posição relativa
		\begin{equation} \label{eq:escore_z}
			a' = \frac{a - \bar{a}}{\delta_a}
		\end{equation}
	\item \underline{Escalonamento decimal:}
		Estabelecido pelo escalonamento decimal move a casa decimal dos valores do atributo a. O número de casas decimais movidas depende do valor máximo absoluto do atributo a.
		A Equação abaixo, na qual j é o menor inteiro tal que $max(|a'|) < 1$, ilustra o cálculo do valor normalizado.
		\begin{equation} \label{eq:decimal}
			a' = \frac{a}{10^j}
		\end{equation}
	\item \underline{Range interquatil:}
		Participa das medidas de posição relativa.
		\begin{equation} \label{eq:interquatil}
			IQR = Q_3 - Q_1
		\end{equation}
	\item \underline{Trivial:}
		\begin{equation} \label{eq:trivial}
			a' = \frac{a}{max_a}
		\end{equation}
\end{itemize}
