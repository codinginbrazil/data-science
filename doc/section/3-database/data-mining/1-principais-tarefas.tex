\subsection{Principais tarefas da mineração de dados}

Objetivo em especificar os tipos de informação a serem obtidas por intermédio das tarefas de mineração, sendo classificada em \textit{descritivas} e \textit{preditivas}, respectivamente, caracterizem as propriedades gerais dos dados; e fazem inferência a partir dos dados analisados.

\underline{Análise descritiva dos dados}
As análises descritivas permitem uma sumarização e compreensão dos objetos da base e seus atributos.

\underline{Preditição: classificação e estimação}
terminologia usada para se referir à construção e ao uso de um modelo para avaliar a classe de um objeto não rotulado ou para estimar o valor de um ou mais atributos de dado objeto.
No primeiro caso, denominamos a tarefa de classificação e, no segundo, denominamos de regressão (em estatística) ou simplesmente estimação.
Sob essa perspectiva, classificação e estimação constituem os dois principais tipos de problemas de predição, sendo que a classificação é usado para predizer \textit{valores discretos}, ao passo que a estimação é usado para predizer \textit{valores contínuos}.

\underline{Agrupamento}

\underline{Análise de Associação}
Existem dois aspectos centrais na mineração de regras de associação:  a proposição ou \textit{construção} eficiente das regras de associação e a quantificação da \textit{significância} das regras propostas.
Ou seja, um bom algoritmo de mineração de regras de associação precisa ser capaz de propor associações entre itens que sejam estatisticamente relevantes para o universo representado pela
base de dados.
